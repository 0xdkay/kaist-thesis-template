\chapter{Exploiting Hidden Data Channels}
\label{sec:accounting}

From the analysis of the accounting policy and VoLTE call flow in
\autoref{sec:anal_data}, we showed the possibility of hidden channels
that an adversary can utilize to bypass accounting. These channels are classified
into channels that reside in the VoLTE call flow (i.e. SIP and RTP tunneling) and a
direct communication channel utilizing a VoLTE default bearer.

\begin{figure}[b!]
  \centering
  \includegraphics[width=100mm]{images/SIP_media_tunneling4}
  \caption{Flow of (1) SIP tunneling and (2) RTP tunneling}
  \label{fig:sip_media_tunnel}
  \includegraphics[width=100mm]{images/direct_tunneling4}
  \caption{Flow of direct communication channel}
  \label{fig:direct}
\end{figure}



\PP{SIP/RTP tunneling} are potentially free
channels in \vt call service, as in~\autoref{fig:sip_media_tunnel}. \CC{1}
In SIP tunneling, the payload is embedded in SIP messages, and these
messages are sent through the default bearer. Meanwhile, \CC{2} RTP tunneling
carries the payload through the established dedicated bearer, as
explained in~\autoref{sec:volte_network}.
Strictly speaking, any protocol can be used for data delivery through the media
session. However, we utilize RTP tunneling as all the operators we tested
encapsulated voice data with RTP.
Note that the voice data transmission using this channel is handled in the CP, and most of the details on its implementation
remain proprietary.

\PP{Direct communication} is another channel that directly sends one UE's
data to another UE.~\autoref{fig:direct} illustrates the flow of direct
communication: \CC{1} phone-to-Internet and \CC{2} phone-to-phone. Since the default
bearer for VoLTE signaling messages is always established as long as the device
is turned on, a UE can easily send data through this bearer to the Internet or another UE unless
P-GW blocks it.

\section{Exploitation}
\label{sec:impl}


We implemented our own sending and receiving modules to verify the hidden
channels in operational networks.~\autoref{fig:module_image} illustrates the sending
module (left) and receiving module (right), which are connected through the IMS
network.

\begin{figure*}[h]
  \centering
  \includegraphics[width=160mm]{images/module_image4}
  \caption{Diagram of Sending and Receiving Module}
  \label{fig:module_image}
\end{figure*}



As 3GPP specifications give some freedom to operators, and it is not
clear if all operators follow 3GPP specifications~\cite{3gpp_ims,
  gsma_volte} completely, our modules take this
implementation-specific deviance into account. For example, for the
operator using IPsec, we utilized the established IPsec tunnel to send
our data instead of sending SIP messages directly.  In the case of direct
communication, we do not require any additional implementation since we
can open a socket at each side and transfer packets directly, if
possible.

\subsection{Sending Module}
\label{sec:sending_module}

First, our sending module should have more functionality than
the native \vt calling application in a mobile phone; it should be
able to vary its parameters such as the sender's phone number, IP address,
and port number to create arbitrary media sessions. The sending module
consists of SIP Parser, SIP Handler, SIP Tunneling Sender and RTP Tunneling Sender.

\PP{\CC{1} SIP Parser} extracts common headers and carrier-specific headers in the
packets obtained from native \vt apps.  For example,
an {\tt INVITE} message contains the caller's phone number, IP address, and routing
information such as IP addresses of SIP servers.  SIP Parser
automatically parses this information and stores it in its database
separated by each operator for later recreation of the SIP message in our
sending module.

\PP{\CC{2} SIP Handler} manages the exploitation. When we initiate our
attack, it takes configuration values: operator's name, phone number,
and phones' IP address and port number for tunneling.  By simply
modifying these configuration values, the SIP Handler can generate SIP
messages for each operator.  It also randomly generates
parameters (e.g. branch, tag, and calling ID) that distinguish each call
session to guarantee freshness.  SIP Handler triggers either
SIP Tunneling Sender or RTP Tunneling Sender for each test.

\PP{\CC{3} SIP Tunneling Sender} establishes a SIP tunnel when it
receives a signal from SIP Handler. It first fragments a file to be
transferred into several blocks.  Then it embeds the fragmented blocks
inside the SIP messages.  Since there exists a maximum number that we
can fragment (otherwise blocked by SIP servers in IMS) and the size of
each block cannot exceed the MTU, the SIP Handler may need to fragment the
file into multiple blocks.  For convenience, we place the data block
at the end of the body in {\tt INVITE}.

\PP{\CC{4} RTP Tunneling Sender} is more complicated than
SIP tunneling.  In the case of RTP tunneling, fragmentation of a given file
is the same as in SIP tunneling. However, we first must establish
a media session to transfer data. Therefore, the RTP Tunneling Sender
generates an {\tt INVITE} and follows the native call flow until it gets an
{\tt OK (200)} message from the callee. It then extracts the IP
address and the port number of the callee from the established media
channel. It transfers the data blocks wrapped as an RTP packet to the
extracted IP address and the port number. In native calling apps,
voice is wrapped and sent from the CP. However, as described in~\autoref{sec:impl_anal}, 
we discovered that audio
packets from the AP to the receiver are routed correctly as well. To
distinguish our packets from others, we add an identifier at the
beginning of the payload. We also add the sequence number and
timestamp after the identifier for our performance evaluation.



\subsection{Receiving Module}
\label{sec:receiving_module}


The receiving module receives the data blocks in RTP packets sent from the
sending module through the IMS network. Our receiving module consists
of the SIP Tunneling Receiver, RTP Tunneling Receiver, Measuring Engine,
and Data Storage.

\PP{\CB{1} SIP Tunneling Receiver} parses SIP messages and extracts
our data from received RTP packets.  Since SIP messages are processed
in the AP, as described in~\autoref{sec:volte_network}, the SIP Tunneling
Receiver can capture incoming packets and parse them in real time. As
we placed the data blocks at the end of the body in the {\tt INVITE}, the
receiver can easily extract the fragmented blocks and reassemble
them. The receiving module opens a raw socket to capture packets
because a SIP daemon is already running on the device.  For the
operator using IPsec, we could easily extract SIP messages out of ESP
packets since we changed the encryption algorithm to Null.


\PP{\CB{2} DIAG} is QualComm's proprietary diagnostic protocol.
It has a command that can be used to mirror every received packet to
the RTP Tunneling Receiver via DIAG interface in the Android kernel
as introduced by Delugre~\cite{delugre2011reverse}. To initiate
mirroring from CP to the DIAG interface, the mobile device has to be
connected to a laptop once. After this, the laptop can be
disconnected.

This step is necessary since the data we sent through the media
channel are only processed in CP, but not forwarded to AP. Therefore,
in order to receive and process packets directly, we need to utilize
the DIAG command.

In addition to the DIAG command, one may consider Android radio
interface layer (RIL) to receive audio data. The problem with the RIL
interface is that some mobile devices do not export incoming voice to
 the AP. Instead, it transfers incoming voice directly to a speaker. Because
of this limitation, we chose to use the DIAG command.

\PP{\CB{3} RTP Tunneling Receiver} utilizes the DIAG interface.  After
the mobile device receives the DIAG command, RTP Tunneling Receiver
starts receiving all network packets through the DIAG interface in the
Android kernel. If the received packet is not corrupted and contains
the identifier we set in the sending module, it accepts the
packet. Finally, it extracts and sends data blocks to Data Storage
while a sequence number and timestamp are passed to the Measuring Engine.

\PP{\CB{4} Measuring Engine} receives a sequence number, timestamp, and data size
from the receivers to evaluate the network performance of our tunneling.
Note that we did not measure the performance for the SIP tunneling
since it might cause denial of service to the SIP servers in the IMS
network (See~\autoref{sec:attack} for more details.)


\subsection{Challenges and Limitations}
During our implementation, we encountered some challenges. First,
many operators do not follow the specifications~\cite{3gpp_ims,
  gsma_volte} for either the flow or the structure of SIP messages. For
example, some operators do not transfer {\tt RINGING} or {\tt SESSION
  PROGRESS} during call setup. Some operators even simply modify or remove header fields for their own purposes.
Consequently, much work had to be done to ensure our sending module adjusts operators' individual VoLTE features and obtain results thereupon.

The second challenge is that at the receiver's side, the device automatically closes the calling session by
sending a {\tt BYE} message when it
does not receive RTP packets for a certain period (typically 10 seconds).
Therefore, we had to wrap the data blocks into RTP packets to keep
the session alive.

Finally, the receiving module requires a mobile device to be connected with a
laptop once to send the DIAG command. However, once logging setup is
complete, the device does not need to be connected with the laptop
until it is powered off. We discovered that the DIAG interface in the Android
kernel does not accept DIAG commands.
 We tried to send the DIAG command from the kernel to eliminate the one time connection to a
laptop, but it was not successful.
In contrast, from our laptop, we could send DIAG commands through the USB connection.
There might be a protecting mechanisms in the CP that blocks DIAG commands from the
mobile device since they are usually sent from control software in a laptop.


\section{Evaluation}
\label{sec:eval}


\subsection{Media Channel Properties}
\label{sec:chann_charac}
We first measured the characteristics of the media channel during the call as in
\autoref{table:media_channel_char}.  The experiment is conducted with
OPTis-S\footnote{OPTis-S is a software that enables
mobile device manufacturers to analyze control-plane messages.} software from
Innowireless. These media channel characteristics represent bearer information
set by operators as well as the QoS parameter for bandwidth designated in the
body of an {\tt INVITE} message.

\begin{table}[h]
  \caption{Media channel characteristics}
  \label{table:media_channel_char}
  \renewcommand{\arraystretch}{1.4}
  \renewcommand{\tabcolsep}{1.2mm}
  \centering
  \begin{tabular}{l| c c c c c c}
    \hline
    & \bf{US-1} & \bf{US-2} & \bf{KR-1} & \bf{KR-2} & \bf{KR-3} \\
    \hline\hline
    Qos Param. (Kbps) & 38 & 49 & 41 & 41 & 49 \\
    Bandwidth (Kbps) & 38/49 & 49 & 65 & 65 & 65  \\
    Latency (sec) & 0.1& 0.1 & 0.1& 0.1 & 0.1 \\
    Loss rate (\%) & 1& 1& 1& 1 & 1  \\
    \hline
  \end{tabular}
\end{table}

When a mobile device establishes a media channel, the network sends a request for bearer creation with QoS information.
We analyzed this request and extracted bandwidth, latency, and loss rate for the media channel. However, some operators do not specify this
information in the message. In this case, we use the QoS class identifier (QCI) value in the message to identify the channel characteristics, as described in~\cite{3gpp_23203}.
In most operators, as shown in~\autoref{table:media_channel_char}, the bandwidth specified in the bearer request is different from the {\tt INVITE} message.


\subsection{Hidden Data Channel Measurements}
We measured the network performance with the sending and receiving
modules for the hidden data channel.  The experiment was conducted on
the same five operators as in~\autoref{sec:anal}.  The feasibility of
accounting bypass in each hidden data channel is shown in
\autoref{table:free}. The table indicates that if we can send data
through a certain channel, it is not charged. In the case of
SIP tunneling and RTP tunneling, all operators are open to free data
transfer.  However, the result of direct communication is different
among operators.  In case of US-1, for example, phone-to-phone
communication is available while phone-to-internet access is
prohibited. The triangle mark in KR-3 means that a free data channel is
available for IPv4, but not for IPv6. Since
direct communication originates from implementation flaws or
P-GW blocking policy, it can vary among operators. Through a feasibility analysis, we
found that phone-to-phone direct communication is available
for operators that do not have a media proxy.

%% Feasibility of each accounting bypass attack
\begin{table}[h]
  \caption{Feasibility of our accounting bypass attacks in each operator}
  \label{table:free}
  \renewcommand{\arraystretch}{1.4}
  \renewcommand{\tabcolsep}{1.2mm}
  \centering
  \small
  \begin{tabular}{l | l | c c c c c c}
    \hline
    & \bf{Hidden Channel} & \bf{US-1} & \bf{US-2} & \bf{KR-1} & \bf{KR-2} & \bf{KR-3} \\
    \hline\hline
    VoLTE & SIP Tunneling   & \cc& \cc & \cc & \cc & \cc \\
    \cline{2-8}
    Call Service                & RTP Tunneling & \cc& \cc & \cc & \cc & \cc  \\
    \hline
    Direct & Phone to Phone    & \cc & \xx & \cc & \xx & \xx \\
    \cline{2-8}
    Communication                       & Phone to Internet & \xx & \cc & \cc & \xx & △ \\
    \hline
  \end{tabular}
\end{table}



We also measured the actual network performance for each operators,
which includes throughput, latency, and loss rate, as shown in
\autoref{table:media_tunneling}. While the information of network
performance is included in the bearer creation request, we conducted this
experiment to derive the actual performance. Since transferred data are
wrapped with a RTP header upon UDP, we added an additional header
containing an identifier, sequence number, and timestamp.  To measure
throughput, we calculated received packet bytes per unit time.  For
latency, the receiver computes the time difference and delay with the first
two packets to sync its time with the sender.

\begin{table}[h]
  \caption{Measurement results of RTP tunneling in each target operator.}
  \label{table:media_tunneling}
  \renewcommand{\arraystretch}{1.4}
  \renewcommand{\tabcolsep}{1.2mm}
  \centering
  \begin{tabular}{l| c c c c c}
    \hline
    & \bf{US-1} & \bf{US-2} & \bf{KR-1} & \bf{KR-2} & \bf{KR-3} \\
    \hline\hline
    Throughput (Kbps) & 37.90& 36.93& 45.76& 39 & 50.48\\
    Latency (sec) & 0.52& 0.02 & 0.10& 0.32 & 0.30\\
    Loss rate (\%) & 1.44& 1.74& 0.77& 0.65 & 0.73 \\
    \hline
  \end{tabular}
\end{table}



Note that in our hidden data channel, we can send data as fast as
possible. However, since the bandwidth is limited, more packets will
be dropped when we increase the throughput. Therefore, by fitting the
loss rate to 1\% (i.e. using the same loss rate as in
\autoref{table:media_channel_char}) by delaying or varying the payload size, we
can obtain the actual throughput.

For the experiment, we sent 200,000 packets and averaged the results.
Since multiple SIP messages can damage SIP servers in the IMS network,
we did not measure the performance for SIP tunneling.  As can be seen
in~\autoref{table:media_tunneling}, the results are different from the media channel
characteristics in~\autoref{sec:chann_charac}. The discrepancy may originate from several factors: number of users, cellular network
status, or signal strength to the cell tower.


We also measured the performance of direct communication: phone-to-phone and
phone-to-internet. The best result of phone-to-phone communication was 16.84 Mbps
in one of the Korean operators. In the case of phone-to-internet, the best result was 21.55 Mbps for the
same operator. This high throughput comes from the bandwidth of
the default bearer for VoLTE signaling being configured in the same manner as the default
bearer for the data service.
However, as described in~\autoref{sec:volte_network}, data transmission through
the default bearer for VoLTE signaling has the highest priority. Therefore, if
a malicious user utilizes the VoLTE default bearer for data transmission, she
will be guaranteed higher performance than normal users.



