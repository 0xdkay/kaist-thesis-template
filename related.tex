% Related work
\chapter{Related Work}
\label{sec:rel}

\PP{Accounting Issues in Cellular Network.}
Several research groups have studied cellular accounting
issues~\cite{peng2014real, go2014gaining, go2013towards,
peng2012mobile,tu2013accounting, peng2012can}. There are two main attacks
related to accounting: accounting bypass and overbilling.

Peng et al. demonstrated DNS port abuse for free-data~\cite{peng2012mobile}.
In~\cite{peng2014real}, the authors uncovered an accounting bypass by source IP
address spoofing because a mobile data charging system was only based on the
packet header. Go et al. wrapped their payload in a TCP retransmission packet
and utilized some ISPs that did not charge a fee to ensure fairness~\cite{go2014gaining, go2013towards}.

In addition to accounting bypass, overbilling attacks, also occur. Go et al. pointed out that
TCP retransmission could also be used to impose a data fee on a
victim. In~\cite{peng2012mobile}, a large quantity of spam data was sent to a victim using
VoIP and a malicious phishing link. The unfair billing practices of some operators were illuminated in~\cite{tu2013accounting, peng2012can}.

While much research has been conducted on accounting bypass and overbilling, our work
is fundamentally different in terms of the interface we used. Previous works
only covered accounting issues related to the data interface, whereas our work is focused
on the VoLTE interface. Furthermore, previous research studies assumed that the overbilling attacks were
committed by first installing  a malicious application on the victim's phone.
In contrast, in our call spoofing attack, the adversary only needs her own device.

\PP{VoIP Tunneling for Censorship Resistance.}
Much research has been conducted on VoIP tunneling, with a specific focus on avoidance of censorship.
In general, VoIP services such as Skype are widely used as  tunneling protocols. Skypemorph sends Tor traffic through UDP ports
via the Skype video call channel~\cite{mohajeri2012skypemorph}.
In~\cite{wang2012censorspoofer}, the authors utilized RTP downstream to avoid
censorship.  In the case of Freewave, data are converted to acoustic
signal data and loaded into normal VoIP packets to hide the data~\cite{houmansadr2013want}.

Our work also utilizes a tunneling protocol and data concealment. However, our focus is on
accounting bypass as well as discovery of mis-implementation problems. In other
words, the main goal of our work for tunneling is different from that of previous works.
In fact, in contrast to previous works, which applied tunneling on the
Internet, our work is the first to apply tunneling in cellular
networks.

\PP{DoS Attack on Cellular Network.}
%Attacks on the cellular network have been researched, too.
Various DoS attacks in cellular networks have also been
investigated.  In general, most research on DoS attacks has been
related to the GSM Network ~\cite{enck2005exploiting, traynor2009cellular,
  mulliner2011sms, qian2012you, golde2013let}.  Enck et al. suggested
that sending SMS to certain phone numbers compiled by a hit-list
would massively affect the cellular core
network~\cite{enck2005exploiting}.  Traynor
et al.~\cite{traynor2009cellular} demonstrated that degrading the cellular
network service was possible with a cellular botnet.  Mulliner
et al.~\cite{mulliner2011sms} described how malformed SMS messages could
force mobile phones to be rebooted, and would finally overload the
network.  Golde et al.~\cite{golde2013let} introduced a DoS attack
because of mis-authentication in the GSM network.
Traynor et al.~\cite{traynor2007attack} presented DoS attacks exploiting the setup
and teardown process of the radio interface in the GPRS/EDGE network.
The UMTS network is still vulnerable to a targeted DoS.  While Enck et al. utilized a
phone number on the GSM network, Qian et al. made a hit-list by
fingerprinting IP addresses to carry out a targeted DoS on the core
network of UMTS~\cite{qian2012you}.

In contrast to these attacks on GSM networks, the DoS attack proposed in
this paper is against the VoLTE network, and can be initiated by a
significantly small number of mobile devices.

\begin{comment}
\PP{Exploiting vulnerabilities in CP.}
There have been a few studies about reverse-engineering of CP in mobile devices. Miras
exploited CP with a buffer overflow vulnerability in a AT command
parser~\cite{miras2011baseband}. In~\cite{delugre2011reverse}, Delugre
introduced Qualcomm's CP structure and a way to setup a debugging environment
for CP. Weinmann exploited GSM software stack utilizing
openBTS~\cite{weinmann2012baseband, burgess2008openbts}. In addition, Mulliner
et al. performed a fuzz testing with manipulated SMS messages utilizing
openBSC~\cite{mulliner2011sms, welte2008openbsc}. In~\cite{welte2010running},
Welte et al. demonstrated an open source GSM Baseband software implementation,
OsmocomBB.

Previous works mainly studied on CP itself while we focused on the signaling
procedure between a UE and the core network. In addition, call
signaling procedure in LTE networks is handled in AP. Thus, an
adversary can easily manipulate signaling packets without utilizing CP.
This capability opens a large attack surface that has not
been appeared in the legacy networks. Therefore, to reveal potential
vulnerabilities residing in the VoLTE signaling procedure, we analyzed it and
discovered several vulnerabilities.
\end{comment}
