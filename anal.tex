\chapter{Empirical Analysis of VoLTE Services}
\label{sec:anal}
In this section, we describe our empirical security analysis
on the current implementation of \vt services by commercial,
deployed mobile cellular network operators.
%
For the analysis, we first analyzed 3GPP standards related with \vt service,
and made a checklist~\footnote{This checklist consists of more than 60 items for
both control and data plane} of potential vulnerable points in the \vt feature.
%
We examined five major carriers~\footnote{We intentionally anonymized the names
of carriers to protect them until all operators fix the problems we identified.}
in the United States and South Korea, the two countries with the highest LTE
penetration ratio for \vt service~\cite{idate2014in}.  For a quick summary, we
found four free data channels and four critical security issues.

For our security analysis, we do not assume a privileged adversary that has physical access to a core network. Instead, we consider an
adversary who is legitimately subscribed to an LTE network. Therefore,
the adversary considers the actual implementation of cellular networks
as a blackbox.  This implies that the adversary can only access the
public information and infer implementation by analyzing the behaviors
of the \vt service.
Also, the adversary is allowed to have full permission of the mobile
devices to access to raw sockets or the device interface in the
Android kernel. In short, the adversary may have root access to the phone.


\section{Analyzing Hidden Data Channel}
\label{sec:anal_data}
As motivated with the problems outlined above, we conducted a thorough empirical
analysis of current VoLTE services. To determine how accounting on VoLTE could
be bypassed, we first analyzed the 3GPP specifications for the VoLTE protocol.
We then investigated the implementation of target carriers (five carriers) by
inspecting actual traffic between a UE and the cellular core in order to check
if there exists any exploitable vulnerability.


\subsection{Potential Free Data Channels}
\label{sec:potential-free-data}

As mentioned above, the accounting for VoLTE call starts when a SIP server
receives an {\tt OK (200)} from UE-B, the receiver. Thus, if UE-B does not send
the {\tt OK (200)} to the SIP server, then the call is not be charged.

An adversary may also consider either sending data by encapsulating it into SIP
messages (e.g. messages before an {\tt OK (200)} such as an {\tt INVITE}) or
directly sending an {\tt OK (200)} to UE-A, the sender, bypassing the SIP
server. Since each UE already has an IP address for its default bearer for the
\vt service, UE-A can send SIP messages directly to the other UE using this
address if it is not blocked by the LTE gateways.

Furthermore, according to the 3GPP
specification~\cite{3gpp_ims}, the QoS parameter for voice traffic
is specified in the {\tt INVITE} message. Thus, if an adversary could
manipulate this parameter, she would be able to increase the bandwidth for sending
a large amount of data.

Note that the above potentially free data channels could be easily
blocked or detected by a flow analysis at the SIP server. However, what if an
adversary embeds the data in the media session? Detecting this
requires significant implementation effort, as the carrier needs to check
if the data in the media session are voice or not.

In summary, an adversary may try to 1) squeeze data into SIP
packets, 2) send data directly to the receiver, or 3) send data over
the media session.


\begin{table*}[!t]
  \renewcommand{\arraystretch}{1.4}
  \renewcommand{\tabcolsep}{0.9mm}
  \centering
  \caption{Characteristics of VoLTE services on tested carriers}
  \label{table:summary}
  \begin{tabular}{l| c c c c c c }
    \hline
    & \bf{US-1} & \bf{US-2} & ~~~~\bf{KR-1}~~~~ & ~~~~\bf{KR-2}~~~~ & ~~~~\bf{KR-3}~~~~ \\
    \hline\hline
    Network protocol &IPv6& IPv6 + IPSec & IPv4 & IPv4 & IPv6  \\
    \hline
    Transport protocol for SIP & TCP \& UDP & TCP \& UDP & UDP & UDP & UDP  \\
    \hline
    Encryption algorithm for IPSec & - & AES & - & - & -  \\
    \hline
    Capability of changing SIP source port & \cc & \xx & \cc & \cc & \cc \\
    \hline
    Existence of a media proxy & \xx & \cc & \xx & \cc & \cc \\
    \hline
    \thead[l]{Capability of sending random data through media session} & \cc& \cc& \cc& \cc & \cc \\
    \hline
    \thead[l]{Capability of changing QoS parameter specified in {\tt INVITE}} & \xx & \xx & \xx & \xx & \xx \\
    \hline
    Free use of audio data & \cc & \cc & \cc & \cc & \cc \\
    \hline
  \end{tabular}
\end{table*}

\subsection{Empirical Analysis}
\label{sec:impl_anal}

With the knowledge of the signaling protocol of VoLTE, we analyzed the actual call
flow in the top five carriers. The characteristics of the service
of each carrier are summarized in~\autoref{table:summary}. Note that each carrier
supports different smartphones for \vt and still only a limited number
of models support \vt. For the analysis, we used the following four
models: Samsung Galaxy S5, S4, and LG G3. In each experiment, we
used at least two different models among them.


\PP{Transport protocol.}  To send/receive SIP messages, Korean operators use UDP
only, whereas U.S. operators use both UDP and TCP. We discovered that the U.S.
operators send response messages such as {\tt ACK} or {\tt PRACK} using UDP,
while TCP is used for all other SIP messages. One of the U.S. operators protects
SIP messages using IPsec with AES encryption. However, we were able to change
the encryption algorithm of IPsec from AES to Null by modifying configuration
file for SIP in the UE side. It should also be noted that we were able to
analyze call flows in plaintext.  Furthermore, we identified that an IPsec
daemon running on the phone automatically wraps packets with a specific SIP port
into the IPsec tunnel. By utilizing this daemon, we were able to send SIP
messages to a SIP server.

\PP{Changing a SIP source port.}
Since the native SIP client in UE-A is already listening on a pre-defined port,
other custom applications cannot bind to this port. Therefore, we checked if SIP
servers accept other source ports, and found that all operators except one U.S.
operator allow this. No regulation on source ports could be problematic if a
malicious app initiates \vt sessions (see~\autoref{sec:perm} for more details).

\PP{Media proxy.}
We also checked whether a proxy for relaying media data exists. The results of our
analysis indicated that two of the operators do not utilize a media proxy, resulting in
UEs being able to directly transfer media data. In this case, when a UE sends an
{\tt INVITE} message to a SIP server, the UE receives a response message with the
other UE\rq{s} IP address. Therefore, a UE can collect another UE\rq{s} IP
address by randomly sending {\tt INVITE} messages to the SIP server.
Furthermore, if a media proxy exists, it could be used to detect malicious
behaviors over the media session by inspecting packets.

\PP{Sending data through a media session.}
When a call is established, a dedicated bearer is also
created for the media session. Voice packets are then sent directly through this bearer from
CP in UE-A.  However, we were able to send packets through this bearer in AP with the
IP address and port number specified in SIP messages for all operators.
In other words, the Android device does not have proper access control for using
the dedicated bearer for the media session.
Furthermore, transmitted data over the media session were not charged
since operators provide unlimited calls for VoLTE users.


\PP{Manipulating QoS negotiation.}
We investigated whether we could manipulate QoS parameters specified in the {\tt
INVITE} to acquire higher bandwidth. However, we discovered that even if we
changed the QoS parameters in an {\tt INVITE}, the actual QoS level of the
dedicated bearer remained unchanged. In other words, for all the operators, SIP
servers do not consider the QoS parameters in the {\tt INVITE}
(see~\autoref{sec:chann_charac} for details).

\PP{Summary.} As discussed, there may exist multiple hidden data channels in
\vt. In~\autoref{sec:accounting}, we show how we exploit these channels and
which operators are open to these channels.

\section{Security Problems of \vt}
\label{sec:security_prob}
From the analysis of VoLTE call flow, we found multiple security issues that can
be critical to both end users and the cellular operators.

\PP{Permission model mismatch.} This is an
interesting issue we discovered.  In general, mobile OSes that run on mobile
devices (e.g. Android) separate out permissions
to regulate the behavior of each application for security
reasons.  For example, in an Android device, an application should
have the call permission, \textit{android.permission.CALL\_PHONE}, to call
other people. In contrast, we discovered that this permission could be violated
due to the adoption of the VoLTE interface on the mobile device. To verify this, we
developed an Android application that only has the Internet permission,
\textit{android.permission.INTERNET}, which enables the application to
send data to the Internet. By abusing this permission, the application can
send SIP messages directly to the SIP server to call other people. This shows
that the current Android permission model cannot distinguish SIP messages
from data communication. In addition, we also found that the calls initiated
by this application do not leave any trace; \textit{the calling state is not displayed on the
phone}. As a result, a user would not be able to recognize that the phone is now
calling.

\PP{IMS bypassing.} To make a call to another UE, a UE sends an {\tt INVITE}
message to a SIP server. Then, a call session is established between two
UEs, and this session is managed by the SIP server. In cellular networks, direct
communication between two phones is usually blocked by gateways for management
reasons, even if two mobile devices know each other's IP address. However, we
discovered that with two of the operators, sending SIP messages directly from
a UE to another one was possible; furthermore, the call session was successfully
established. As a result, the communication could not be accounted as discussed
in~\autoref{sec:signal_protocol}. In addition, such direct communication can
lead to bypass user authentication of IMS. This problem originates from the
inappropriate access control at LTE gateways on the default bearer for SIP
signaling.


\PP{No authentication.} If a SIP server receives a SIP message, it should should
authenticate the SIP message to check if the sender is actually a valid user.
However, we discovered that two of the Korean operators do not perform proper
user authentication.  As a result, we could make a call with a fake phone number
by sending a manipulated {\tt INVITE}.

\PP{No session management.}
In addition, none of the operators except one Korean operator appears to manage
call sessions correctly. As a result of this incorrect session
management in SIP servers, an adversary is able to make phone calls
to many people simultaneously. To establish a call, as described
in~\autoref{sec:signal_protocol}, a dedicated bearer should be established in
advance, even if the receiver does not respond to the call, and the cost of this
establishment is quite expensive. Therefore, if compromised phones start sending
multiple {\tt INVITE} messages which generate a number of bearers, this will
deplete the resources in the core network.



\PP{Summary.} \vt potentially has several security problems as we discussed in
this section. In~\autoref{sec:attack}, we discuss how these vulnerabilities can
be exploited among different operators.

