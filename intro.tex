% 1 Introduction
\chapter{Introduction}
\label{sec:intro}

\section{Motivation}

% popular \vt
Due to the increasing demand for data-centric services,
mobile network operators are quickly moving towards high-speed networks.
%
With higher bandwidth and lower latency, Long Term Evolution (LTE) has
become the dominant cellular network technology in recent years.
%
One distinctive feature of LTE is the way it delivers data such as voice
and SMS; LTE operates through packet-based switching, whereas traditional
cellular networks (e.g., 2G or 3G) rely on circuit-based switching for
their voice service.
%
To reliably serve voice calls on a packet-switching only network, 
mobile network operators have adopted, deployed
and recently started a service for end-users, called Voice-over-LTE
(\vt)~\cite{gsma2014volte}, which is similar to a Voice-over-IP
(VoIP) service in spirit.

% lack of research in to-be-popular \vt
Today, mobile network operators are aggressively deploying \vt services:
by April 2015, 16 operators in 7 countries
had commercially launched VoLTE services, 
and 90 operators in 47 countries are
investing to deploy VoLTE services in the near future~\cite{global2015evolution}.
%
Despite this fast-moving trend,
little research has been conducted to
systematically examine security issues
in the upcoming VoLTE services,
not only in terms of their end-facing interfaces but also their cellular
infrastructure.


% potential issues
The use of packet-switching in \vt opens a large
attack surface that has not been seriously considered thus far.
%
In circuit-switching mobile networks, the signal processing is conducted by a
communication processor (CP) in a mobile phone, whose detailed implementation
is proprietary to a few chip manufacturers.
%
On the contrary, \vt-compatible devices perform signal processing
purely in their application processor (AP):
e.g., initiating Session Initiation Protocol (SIP)~\cite{rosenberg2002sip}.
%
As with general-purpose computers,
\vt relies solely on Internet Protocol (IP) for packet delivery.
Consequently, well-known offensive techniques
targeting the IP are also applicable for abusing or attacking \vt-based devices.

% unsurprising issues we know
Several studies exploring potential attacks and countermeasures in SIP and VoIP services have been conducted, including, for example,
works on breaking authentication~\cite{beekman2013breaking},
bypassing accounting~\cite{zhang2007billing},
mounting man-in-the-middle attacks~\cite{zhang2009feasibility},
introducing various attacks by a hacker~\cite{ozavci2013voip},
and even standardizing basic security issues such as
credentiality, integrity and authenticity, by communities~\cite{3gpp_access_sec, 3gpp_net_sec, arkko2003security, wang2008ims}.

% surprising issues in \vt
Since \vt operates on the cellular network,
it is not only exposed to VoIP-related issues,
but also inherits security issues from the cellular network, such
as adversaries being able to interpose the signal processing itself.
%
For example, unlike VoIP services, the LTE network provides a communication
channel (which is called a {\it bearer}) with guaranteed bandwidth, once
a VoLTE call is established. As it becomes easy to interpose call
signaling with VoLTE functionality, an adversary can create and
utilize private communication channel for peer-to-peer data exchange,
which is not a supported feature in cellular networks. Furthermore,
since most operators do not charge the use of a dedicated channel for
VoLTE service, an adversary can utilize it without being charged.


More seriously, its implementation caveats in authentication and
session management for \vt make its infrastructure vulnerable, and,
therefore, an adversary can easily mount various attacks that bypass
the security policy of \vt.

Regarding problems with user equipment (UE), VoLTE opens a
security loophole whereby an adversary can make a call without suitable
permission for voice, since the current permission model for Android
devices is only suitable for legacy circuit-switching calls.


\section{Research Direction and Contribution}
In this paper\footnote{Most of this thesis is originated from our CCS
paper~\cite{kim2015breaking}.}, we first present problems of commercially
deployed VoLTE services in 5 operators in the United States and Korea
\footnote{We performed experiments on 6 operators, but we failed to connect to
one operator's \vt service, unlike its advertised service coverage.}.
%
These problems are mainly caused by legacy policies and the immature software
infrastructure of VoLTE.
%
To show the motivation for addressing these problems, we demonstrate various attacks that
(1) piggyback a hidden, free data channel
(e.g., free extra bandwidth to the adversary),
 (2) bypass \vt's accounting system (e.g., direct calling bypassing the charging server),
and (3) abuse the VoLTE service
(e.g., caller spoofing, overbilling attack).

In addition, we propose immediate and potential countermeasures for these
problems. For immediate solutions, 1) operators may deploy DPI (Deep Packet
Inspection) for detecting a hidden data channel, 2) strict session management on
both the phone and operator side is required, and 3) cellular gateways have to
be fixed to prevent hidden channels. As a longer term solution, we suggest
changing the accounting policy of operators, and tighter security implementation
at the mobile devices.  To completely eliminate problems, however, cellular
operators, device manufacturers, and mobile platform providers must draw up a
comprehensive solution.

To summarize, we make the three following contributions:
\begin{itemize}
\item To the best of our knowledge, this is the first attempt
  to analyze the security loopholes
  of commercially deployed \vt services.
  We found three previously unknown security issues on VoLTE:
  1) hidden data channels in VoLTE services,
  2) mis-implementation of the cellular operators, and
  3) fundamental problems in the mobile devices.
\item To address these problems, 
  we successfully demonstrate an attack that enables a free data
  channel, and various abuse attacks
  including call spoofing and denial-of-service.
  At the time of the submission, all bugs and exploits are reported to
  operators.
\item We proposed effective, immediate countermeasures,
  and further devise a long-term and comprehensive solution
  that can eliminate current security issues in the VoLTE service.
\end{itemize}

\section{Thesis Structure}
The rest of the paper is organized as follows. ~\autoref{sec:back} presents an
overview of the VoLTE system with the network architecture and call setup procedure
including the accounting policy of real operators.  In~\autoref{sec:anal}, we
introduce current problems and threats of deployed VoLTE services through an analysis
of VoLTE call flow.  We explain the details of hidden data channel attacks and
implementation details along with the measurement results in
\autoref{sec:accounting}. ~\autoref{sec:attack} describes possible attacks
caused by several implementation flaws in the VoLTE service.  In~\autoref{sec:cm},
we provide countermeasures of our attacks and discuss immediate and
fundamental solutions. ~\autoref{sec:rel} includes related work, and we
conclude our study and present directions for future work in~\autoref{sec:concl}.

