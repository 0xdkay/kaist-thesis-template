% 5. attack
\chapter{Exploiting VoLTE Security Problems}
\label{sec:attack}

In~\autoref{sec:security_prob}, we described several security problems
that an adversary can exploit to carry out malicious
behaviors/activities.  In this section, we discuss possible attacks
that could be exploited using the discovered vulnerabilities.
\autoref{table:attack_summary} describes the vulnerabilities with
possible attacks and discusses if they are exploitable in each operator.

\begin{table*}[h]
\renewcommand{\arraystretch}{1.4}
\renewcommand{\tabcolsep}{1.0mm}
\centering
\caption{Vulnerabilities and possible attacks in each operator}
\label{table:attack_summary}
\begin{tabular}{|c|l|c|c|c|c|c|l|}

\hline
\bf{Point}     & \bf{Vulnerability}       & \bf{US-1} & \bf{US-2} & \bf{KR-1} & \bf{KR-2} & \bf{KR-3}  & \bf{Possible Attack}  \\
\hline\hline
UE                          &Permission Mismatch & \multicolumn{5}{c|}{Vulnerable for all Android devices} & Denial of Service on Call, Overbilling \\
\hline
P-GW            & IMS Bypassing & \cc & \xx & \cc & \xx & \xx      & Free Video Call, Caller Spoofing \\
\hline
\multirow{2}{*}{IMS}  & No Authentication & \xx & \xx & \cc & \cc & \xx           & Caller Spoofing \\
\cline{2-8}
               &  No Session Management      & \cc & \cc & \cc & \xx & \cc & Denial of Service on Core, Cellular P2P \\
\hline
\end{tabular}
\end{table*}




\section{Permission Model Mismatch}
\label{sec:perm}
In our experiment, an application with only Internet access permission can make
a call. In addition, this calling activity is not displayed on the screen; thus, a user
may not know that her device was making a call. Consequently, if a malicious application is installed
on a victim's device, an adversary can conduct a couple of attacks by exploiting this vulnerability.

\PP{Denial of Service on Call} is an easy way to block a victim's phone.  With a
malicious application installed on the victim's phone, an attacker can cause the
phone to make calls to designated number(s) repeatedly. This activity would result in 
the victim not being able to receive any incoming calls. 
Furthermore, because the call is not displayed on the screen, the
victim would not be cognizant that her phone is in a calling state. Therefore, this can
cause denial of service on calls to the victim.

\PP{Overbilling} is another powerful attack. If a malicious application
installed on a victim's phone can send an {\tt INVITE} message that initiates
an expensive video call, severe overbilling of the victim can occur.

The permission model mismatch problem shows that the current
permission model used by mobile phones cannot handle the All-IP
environment properly. In the case of 3G networks using a circuit-switching network for
voice calls, call permission and data permission are completely
separated.

\section{IMS Bypassing}
IMS bypassing is a security problem originating from the policy
configured in P-GW.  Even though direct phone-to-phone communication should be
blocked because it can allow attacks such as overbilling attacks, some operators
do not prevent this access in VoLTE.

\PP{Free Video Call} is a useful application of direct communication.  Because
direct phone-to-phone communication bypasses the IMS network, and only goes
through P-GW, one can directly send an {\tt INVITE} message for a video call to another party.
 On the callee's side, the phone only replies to the source IP address of
the received {\tt INVITE}, and no SIP server is involved in the procedure.  Further, because all
accounting related to VoLTE calls is handled in the IMS network, by using this method one can talk to
others without being charged. One requirement is that the sender has to open a
microphone, but this does not pose a difficulty because the user can simply root the
phone. In the case of Korea, video calls from operators cost 1.66 times
the price of a voice call. In the case of the U.S., the operators charge
for both data and voice.

\PP{Caller Spoofing} is a severe issue related to bypassing IMS. Because
packets are only routed through P-GW, there is no authentication between the
caller and the callee.  Accordingly, one can send a manipulated {\tt INVITE} to
spoof the victim.  If the adversary modifies the phone number in the {\tt INVITE},
the modified number will be on the screen of the victim's phone. Therefore, the victim
would believe that the call is from that number.  As a result, an adversary can
exploit direct phone-to-phone communication for voice phishing by simply changing a
few bytes in the {\tt INVITE}.


\section{Lack of Authentication}
Absence of authentication is one of the threats originating from
mis-implementation of the IMS network.

\PP{Caller Spoofing} is also feasible even when SIP messages go through SIP
servers. When an adversary generates a modified {\tt INVITE} message and sends it to the
SIP server, if the SIP server only checks if the phone number is valid, the
server is vulnerable to caller spoofing.  When this is successful, it can
also cause a calling fee to be imposed on the person who owns the modified phone number.
Therefore, this attack can be considered a simple yet powerful attack. In fact, not 
only {\tt INVITE}, but also a {\tt BYE} message can be used for caller spoofing.  
If operators do not properly authenticate users, {\tt BYE} messages can be transmitted
to terminate the victim's on-going call. 

We found that two operators in Korea are vulnerable to caller
spoofing. Other operators prevent spoofing using either of the
following two methods: verifying the caller's phone number with the IP
address or with International Mobile Station Equipment Identity (IMEI),
a unique identifier for mobile
devices.


\section{Lack of Session Management}
Absence of session management is another issue of the IMS network. Since some
operators do not manage call sessions, one can send multiple {\tt
INVITE} messages to the SIP server.

\PP{Denial of Service on Core network} is a possible attack in
this case.  When the SIP server receives an {\tt INVITE} message, it
should open a session for each message and manage each session
independently. If the number of {\tt INVITE}s are too large, this can
damage the SIP server and paralyze the IMS network for VoLTE service.

In general, a user can only call once at a time with the native
calling app in the mobile phone. However, with our sending module, we
can transmit virtually an unlimited number of {\tt INVITE} messages.  In
our analysis, if UE-A sends {\tt INVITE}, the dedicated bearers among
UE-A, the P-GW, and UE-B are all established, even when UE-B
does not answer the call. Since the cost of the bearer activation and
release procedure is expensive among control-plane procedures,
multiple {\tt INVITE}s can overload the P-GW, causing a denial of service 
on the core network.

We conducted our experiment only sending 2-4 {\tt INVITE} messages,
and checked whether a call session for each message is created. We found that except for one operator in Korea, all the other
operators in the experiment are vulnerable. The most important aspect of
 our attack is that we can commit a denial of service attack with only
one mobile device whereas such an attack usually requires a
huge number of bots. Of course, we were not able to verify the
cost of each bearer in the operator, or whether this attack actually
shuts down the SIP server.




%should revise from here
\PP{Cellular P2P} is a more complicated application but is still feasible.
Since there is no session management, users can send multiple INVITE messages to create several call sessions.
When multiple call sessions are established among users, people can share files as torrents through RTP tunneling.
Even though the 1\% loss rate is high, we can utilize the reliable UDP protocol, as in~\cite{bova1999reliable}. The throughput
is adequate and there is still enough speed to transfer files as a torrent because there are many peers in the cellular network, and these peers do not usually shutdown their phones. Therefore, people can share movies or other content when they are sleeping.
If implementation is seriously concerned, Cellular Tor can also be available to evade censorship on the cellular networks.

