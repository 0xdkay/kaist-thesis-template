\begin{summary}
LTE가 이동통신망의 주요 기술로 자리잡으면서 이동통신망에는 다양항 변화가
이뤄지고 있다. 기존의 회선 교환 방식으로 제공되었던 통화 방식과 달리 LTE에서는
VoLTE라는 새로운 기술을 도입하여 음성 및 영상 통화를 패킷 교환 방식으로
서비스하게 되었다. 패킷 교환 방식을 적용한 VoLTE 기술은 성능적인 측면에서는 많은
향상을 기대할 수 있지만, 해당 기술을 적용하기 위해서 이동통신망과 사용자
단말에서 음성을 처리하는 방식이 많이 변하게 되었고, 이로 인해서 새로운
문제점들이 나타나게 되었다. 따라서 본 논문에서는 VoLTE와 관련된 보안 문제점들을
밝히고 이를 해결하는 방법들을 제시한다.

VoLTE에서의 통화 방식은 기존의 통화 방식과는 다르게 IP 기반의 SIP를 이용하여
음성 데이터를 전달한다. 망을 거쳐 단말까지 전달된 음성 패킷은 단말의
어플리케이션 프로세서에 의해서 처리가 된다. 따라서 망에 정상적으로 등록되어 있는
사용자 중 단말의 어플리케이션 프로세서를 조작할 수 있는 사람이라면 누구나 다
VoLTE 통화 과정에서 발생하는 문제점들을 찾고 이를 이용하여 공격을 수행할 수
있다. 이러한 상황은 기존의 통신사들의 무제한 음성통화 정책이나 음성과 데이터를
따로 과금하는 정책과 맞물리면서 통신사의 과금을 우회할 수 있는 다양한 공격이
가능하다. 또한 보안에 대해 충분히 고려하지 않고 도입된 VoLTE 기술은 단순한 과금
우회 뿐만 아니라 사용자와 이동통신망 자체에 심각한 영향을 초래할 수 있는 보안
취약점들을 만들어내게 되었다. 본 논문에서는 만약 악의적인 사용자가 이러한
취약점들을 이용한다면 발신번호 조작, 특정 사용자에게 과금 부여, 망에 대한 서비스
거부 공격 등 다양한 공격을 수행할 수 있음을 확인하였고, 발견한 보안 취약점들을
보완할 수 있는 방법을 제시하였다.  하지만 근본적으로 VoLTE의 보안 문제를
해결하는 것은 쉽지 않으며, 이를 위해서는 단말, 모바일 운영체제, 이동통신망
사업자 등 이동통신망과 관련된 모든 주체가 협력하여야 한다.
\\
\\
\\
\textbf{\textit{핵심어}}: VoLTE, 과금, 보안 취약점, 이동통신망

\end{summary}
