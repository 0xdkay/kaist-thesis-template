% 2 Background
\chapter{VoLTE Overview}
\label{sec:back}

\section{LTE Network Infrastructure}
\label{sec:cellular}

A cellular network comprises two architectural components: an access
network that the UE connects to, and a core network that supports its
cellular infrastructure.
 

\section{VoLTE Service}
\label{sec:volte_network}

\vt service is introduced to deliver voice calls
over the packet-switching based LTE network.
The service utilizes an IMS network based on SIP,
similar to VoIP service over the LTE network.
To establish a voice call, a UE
follows standard procedures as depicted in~\autoref{fig:volte_procedure}.

\begin{figure}[h]
  \centering
  \includegraphics[width=130mm]{images/volte_procedure4}
  \caption{Overview of packet-switching and IMS protocols in \vt; registration and call setup
    between a UE and a LTE network.}
  \label{fig:volte_procedure}
\end{figure}


\section{VoLTE Signaling Protocol}
\label{sec:signal_protocol}


\PP{Call Signaling.}
If the user accepts the call at UE-B, it sends an {\tt OK (200)}


\PP{Call Management in the UE.}
speaker and a microphone. Upon receiving voice traffic from a radio
channel, the CP processes the voice packets and forwards only audio data to
the AP, which lessens the computational burden of the AP. 



