% 2 Background
\chapter{VoLTE Overview}
\label{sec:back}

\section{LTE Network Infrastructure}
\label{sec:cellular}

A cellular network comprises of two architectural components: an access
network that the UE connects to, and a core network that supports its
cellular infrastructure.
%
\autoref{fig:cell_arch} illustrates the two-folded architecture of 3G
and LTE networks. The access network (left side) is a radio connection
where the UE accesses a base station (e.g., NodeB in 3G and evolved
NodeB, \mbox{eNodeB} in LTE in short).
%
On the other hand, the core network (gray region) handles
service-level connections such as voice calls and the Internet (e.g.,
PSTN in 3G and IMS~\cite{3gpp_ims} in LTE).
%
IP Multimedia Subsystem (IMS) offers IP-based voice calls and multimedia
services by using the SIP (bottom), and Public Switched Telephone
Network (PSTN) provides a typical public telephone network (top).
%
4G GWs in the core network consist of Serving Gateway (S-GW) and P-GW.  S-GW is
a mobility anchor for inter eNodeB handover and relays the traffic between 2G/3G
systems and P-GW. Meanwhile, the P-GW manages the PDN connection between a UE
and a service.  It is also responsible for packet filtering and charging, which
are crucial functions for preventing accounting bypass and service abuse
attacks.

\begin{figure}[h]
  \centering
  \includegraphics[width=130mm]{images/Cellular_architecture5}
  \caption{Two-folded architecture of 3G and LTE networks. Mobility
    Management Entity (MME) in LTE stands for user mobility.
  }

  \label{fig:cell_arch}
\end{figure}
 
The major difference between 3G (top) and LTE (bottom) networks is in the
way they deliver data in the core network. The 3G network separates 
network domains into packet-switching for the Internet connection and
circuit-switching for phone calls.
%
The mobile switching center (MSC) in the circuit-switching domain transfers voice calls,
and the 3G gateways enable data communication in the packet-switching domain.
%
In contrast, the LTE network only operates through the packet-switching
domain; as it does not have a circuit-switching domain, its voice calls either fall
back into the 3G network (also known as Circuit Switched Fallback, or CSFB
in short) or
%
the LTE provides a \vt solution to transfer both voice calls and data
to the packet-switching domain, which does not require any fallback to the 3G circuit-switching
network.

\section{VoLTE Service}
\label{sec:volte_network}

The \vt service is introduced to deliver voice calls
over the packet-switching based LTE network.
The service utilizes an IMS network based on SIP,
similar to VoIP service over the LTE network.
To establish a voice call, a UE
follows standard procedures as depicted in~\autoref{fig:volte_procedure}.

\begin{figure}[h]
  \centering
  \includegraphics[width=130mm]{images/volte_procedure4}
  \caption{Overview of packet-switching and IMS protocols in \vt; registration and call setup
    between a UE and a LTE network.}
  \label{fig:volte_procedure}
\end{figure}

To connect to the LTE network, \CC{1} a UE first contacts the eNodeB,
and then \CC{2} the UE registers itself to the Evolved Packet System (EPS),
establishing an Internet Protocol (IP) connection, which is
identified by a \emph{default bearer}. Note that this IP address is
different from the one used for data connection. In other words, every
phone supporting \vt is assigned two IP addresses: one for
voice and the other for data~\cite{3gpp_ip}. 
%
Once the
UE has an IP connection to \vt, \CC{3} the UE connects
to the IMS network and the IMS server authenticates whether the device
is allowed for the \vt service.

If authenticated, \CC{4} the UE can make a voice call through
the SIP signaling service,
provided by Call Session Control Function (CSCF) servers.
%
When a call session is established,
\CC{5} a \emph{dedicated bearer}
is created to identify voice-related traffics
and \CC{6} all voice packets are transferred
through this dedicated bearer.
%
\CC{7} Upon call termination, the bearer used for the
voice session is released.

Note that two bearers are used to enable a connection in the \vt service.
%
The default bearer
established during the EPS registration is for call signaling.
Once the default bearer has been established,
all the incoming and outgoing SIP packets are bound to
this bearer.
%
%However, this default bearer has lower priority than the bearer
%for traditional cellular services like voice call.
%
According to the 3GPP specification~\cite{3gpp_23203}, this default bearer
 has the highest priority among all possible bearers for voice or data services.
%
The main reason for this different prioritization is to support the
QoS of a phone call, similar to circuit-switching based routing.
%
When a phone call is made,
the IMS Packet Data Network (PDN) temporarily creates a dedicated bearer
that has a lower priority than the default one.
This dedicated bearer, however, has a higher priority than the bearers for data services.
%to transmit voice data. (But, it is still higher than the bearers for normal data services.)
%
Although the dedicated bearer has the same IMS PDN address as the default
bearer, it operates by different rules that allow voice packets with the
negotiated media port to pass through the dedicated bearer.

\begin{figure}[h]
  \centering
  \includegraphics[width=100mm]{images/invite3}
  \caption{
    VoLTE signaling (SIP) flow of Call setup and tear down
  }
  \label{fig:inv}
\end{figure}


\section{VoLTE Signaling Protocol}
\label{sec:signal_protocol}


\PP{Call Signaling.}
%Once the IMS registration is completed, the UE is ready for processing incoming/outgoing VoLTE calls.
\autoref{fig:inv} illustrates the call setup procedure between two
UEs, UE-A and UE-B. This can be considered as a more detailed version
of \autoref{fig:volte_procedure} from \CC{4} to \CC{7}, corresponding to the 
VoIP protocol. To initiate a VoLTE call to UE-B,
UE-A first generates an {\tt INVITE} message and sends it to a SIP
server~\footnote{In this paper, for simplicity, we use the SIP server to represent all
  Call Session Control Function (CSCF) servers in the IMS network.}. The {\tt INVITE} message contains a description of
the caller's phone number, IP address, and media characteristics: a
port number, encoding scheme, and QoS parameters for media
communication. Upon receiving the {\tt INVITE} message, the SIP server responds to
UE-A with a {\tt TRYING (100)} message, and then it relays {\tt INVITE} to
UE-B after checking if the message is valid. Upon receiving the
{\tt INVITE} message, UE-B responds with {\tt RINGING (180)} and {\tt
  SESSION PROGRESS (183)} to indicate its call session is being
processed. In response to the {\tt SESSION PROGRESS} message, UE-A
sends a progress ack ({\tt PRACK}) message.

If the user accepts the call at UE-B, it sends an {\tt OK (200)}
message containing information similar to the {\tt INVITE}
message. When the SIP server receives the {\tt OK (200)}, it routes this message to UE-A
and starts charging the calling session. As soon as UE-A receives the {\tt OK (200)},
an end-to-end media session for voice data is established between UE-A
and UE-B. This media session contains a dedicated bearer for both UE-A
and UE-B, and a media proxy in the IMS network. Typically, this media
session is implemented using RTP (Real Time Protocol) on top of
UDP. When either of the UEs wants to terminate the call, it sends a {\tt BYE}
message to the SIP server. Upon receiving such a message, the SIP server stops charging the
call session and routes the {\tt BYE} message to the other UE, and terminates the
media proxy.


\PP{Call Management in the UE.}  After a call session is
established, UEs transfer voice data to each other through the
established media channel. The smartphones used for our experiment have
two processors: an application processor (AP) for running the smartphone
operating system (e.g., Android) and user applications; and a
communication processor (CP) for handling radio access and
radio-related signaling.

Call signaling is handled by a SIP client running in the AP. This SIP client
binds its socket on a specific port (default: 5060) to communicate with SIP
servers. Meanwhile, the CP has a digital signal processing (DSP) module that
handles radio communication with base stations as well as audio data from a
speaker and a microphone. Upon receiving voice traffic from a radio
channel, the CP processes the voice packets and forwards only audio data to
the AP, which lessens the computational burden of the AP. 

\section{Mysteries of VoLTE}
\PP{Accounting of VoLTE calls.}
While \vt is implemented on the packet-switching network (i.e. it runs on IP), it
applies a legacy time-based charging policy as in 3G networks.
Traditionally, a voice call delivered through the circuit-switching network is charged according to
the duration of time it occupies the channel (e.g., \$15 for 225
minutes).  In contrast, data connection through a packet-switching network is
charged based on byte-usage (e.g., \$15 for 1GB). As \vt utilizes
only a packet-switching network, it consumes the same byte-usage as for the data
connection. Despite this, it is quite odd that the majority of
carriers still charge the voice service by time duration. This {\it
  discrepancy} could complicate accounting procedures, since different
IP addresses are used for both data and voice, and their accounting
units are different. Furthermore, many operators recently provide
unlimited calls among \vt users as a default pricing plan, which could
be used as a free data channel, if exploited.

\PP{VoLTE solution in device.}
As described in~\autoref{sec:signal_protocol}, the call signaling procedure
in 3G is handled in CP. Because the details of CP is not disclosed to the
public, a malicious application installed on a mobile device cannot easily
manipulate the call signaling. However, in \vt, as the call signaling is handled
in AP, it could open a new attack surface to an adversary that she can directly
manipulate the call signaling and perform malicious behaviors. Therefore, to
diagnose the potential vulnerabilities of newly adopted \vt, we performed
an empirical analysis of \vt services.



