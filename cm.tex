% 6 Countermeasure
\chapter{Countermeasures}
\label{sec:cm}
In this section, we discuss solutions for both free data channels and VoLTE
service abuse attacks. Since these solutions are quite intuitive, they can be
easily applied to the commercial cellular networks.
Some attack vectors originate from mis-implementation of operators,
while others are derived from fundamental problems of the \vt system.
These fundamental problems are more difficult to obviate than the other attacks.
To prevent these attacks, we suggest more difficult yet comprehensive solutions for securing
the overall \vt service.


\section{Immediate Solution}
\PP{Filtering.}
The main cause of direct communication is inappropriate access control of
user-initiated requests at the gateways in cellular networks. The purpose of the
\vt default bearer is call signaling; therefore, the gateways should filter out
all other packets except SIP messages.  In our analysis, however, some operators
do not follow the standard for call related services.  For example, one operator
in Korea provides conference calls using their proprietary protocols on top of
HTTP although an IETF standard~\cite{johnston2006session} provides a conference
call solution using the SIP protocol.  This inconsistency of service
implementation can create difficulties in correctly managing the access control
at the gateways.

In addition, operators should block all packets travelling directly between UEs,
and only allow packets from UEs to the SIP server/media proxies, and vice versa.
As a result, free data channels as well as free video calls and caller spoofing
could be blocked.

\PP{Strict Session Management.}
Proper session management is a basic requirement for the security of any server.
SIP tunneling, denial of service, and cellular p2p are all possible attacks
resulting from the absence of session management at the SIP servers.
If a SIP server carefully inspects SIP messages originating from UEs, it can
block SIP tunneling.  For example, it can check whether an invalid field or
content exist in the header and the payload. If the result of the check is
unsatisfactory, it should reject the requests and respond to UEs with an error
message.

Further, to protect against SIP tunneling and denial of service attacks,
operators should limit the number of SIP messages from a UE within a certain
period. If an adversary exceeds a certain threshold, operators should block the
adversary and inspect her activities.  Furthermore, the SIP server should check
if a UE is already calling another UE. If so, it should block other call
originating messages (i.e. {\tt INVITE} messages). This policy can detect and
prevent other attacks such as cellular p2p definitely, and intuitively because a
prerequisite of these attacks is that an adversary should be able to send
multiple {\tt INVITE} messages.

\PP{UE Verification.}
To prevent call spoofing, SIP servers should verify the source of SIP messages.
This can be achieved by adding a UE's unique identity (e.g., IMEI) to the header
of SIP messages, so that SIP servers can cross-check the phone number with the
unique identity.
Unless a unique identity is stored securely in the mobile device, an adversary can
easily intercept this information by remotely installing a malicious
application.

Another solution is to bind the phone number with the allocated IP address of a UE. The
operators can validate the user-initiated SIP messages by checking UE's IP
header with the parameters (such as IP address, phone number, and device unique
identity) in SIP messages because this information is already stored in the
operators\rq{} server.  IP spoofing is also possible, but we found that all the
operators we experimented with have an IP spoofing prevention mechanism for the
data interface. Even though we have not tested IP spoofing for the \vt
interface, the operators might have already installed a similar mechanism in
\vt. Therefore, cross-checking is a strong yet easy to implement solution.

However, it cannot prevent caller spoofing in cases where an adversary spoofs 
the victim's IMS registration procedure, such that the SIP server stores spoofed
parameters. A UE registers itself to a SIP server by exchanging REGISTER messages,
and the SIP server stores several parameters that can identify the UE. Thus, if
an adversary can spoof the registration, she can obtain the full
permission of the victim's phone address.  

\PP{Deep Packet Inspection (DPI).}
Since the RTP tunneling exploits the media dedicated bearer during a \vt call,
operators can recognize if a user is utilizing the media channel through traffic
monitoring. This monitoring job can be done by applying a deep packet inspection
(DPI) solution in P-GWs or media proxies. However, if an adversary disguises
the data into voice-like traffic, then the DPI solution will not be sufficient to prevent it.

\PP{Accounting Policy.}
All the free channels we exploited in this paper resulted from
time-based accounting. This problem can be resolved by changing the
accounting policy of \vt service to a byte usage-based scheme. Of
course, even in this case, an adversary can still exploit a voice
channel. However, she cannot bypass the accounting. This seems quite
natural, but it would be difficult for the operators to change the
current time-based policy since it is deeply related to the
revenue. 70\% of total revenue in the operators is still from 
voice and SMS~\cite{Ericsson_press}.


\section{A Long-term and Comprehensive Solution}
Permission model mismatch is a severe problem that is prevalent in current
\vt-compatible mobile devices. Unlike the previous call mechanism, as \vt is
IP-based, the current permission model of mobile devices cannot handle it. One
possible solution is force the sockets from applications to use the data
interface.  In this way, SIP packets from an application cannot reach the SIP
server. The operators, in the same manner, should block packets from the data
interface. One limitation of this solution is that deploying the solution would
not be easy since all the firmware of mobile devices should be updated.
Furthermore, data encryption (e.g. IPsec or TLS for signaling, and sRTP for
media data) should be deployed as specified in the 3GPP
specifications~\cite{3gpp_net_sec, 3gpp_access_sec}.

However, even with strict binding and encryption, an adversary can still utilize
tunneling since she has all permission for her phone.  Another way to resolve
the problem is to process both call signaling and voice data transmission in the
CP as a traditional circuit-switching call does.  The CP only allows legitimate
call related requests and blocks all other packets utilizing the \vt interface
from the AP. Thus, an adversary cannot send manipulated packets through the \vt
interface. Furthermore, protecting the CP with hardware security modules such as
TrustZone or secure storage~\cite{alves2004trustzone} may also be required to
prevent the adversary from intercepting SIP messages in the CP.

