% 8 Conclusion
\chapter{Future Work and Conclusion}
\label{sec:concl}

In this paper, we analyzed the VoLTE features of five operators in the United
States and South Korea.  To the best of our knowledge, we are the first to
analyze security problems of VoLTE in the commercial cellular networks.  This
study mainly contributes the overall security of VoLTE system showing that
moving towards VoLTE is not just a simple transition because it involves the EPC
core (3GPP standards), OS support at UE, hardware interface redesign, and shift
on cellular accounting policies.  The main lesson is that it negates the common
belief that VoLTE is a simple transition because VoIP is well understood on the
Internet. In contrast, it shows that architectural aspects of cellular networks
make the problem much more complex.  Although we discovered a few implementation
bugs that are easy to fix, the core problem is complicated processes, involving
accounting, access control, session management, and EPC-UE interaction. This is
evidenced by the response from ISPs, Google Android, and US/KR CERTs to our
responsive disclosure. It requires greater attention because a systematic
security analysis of new architecture is always necessary to make the
architecture robust.

In this paper, we considered security issues and possible attacks related to
VoLTE call service after legitimate IMS registration.  However, an attacker can
also utilize a SIP {\tt REGISTER} message to perform other attacks.  If there
are vulnerabilities in the registration phase, an attacker can control all
access to a victim's VoLTE service.  For example, she can carry out an imposter
attack or even wiretapping.  We plan to investigate scenarios such as this in
future work.  In this work, we concentrated on the problems and vulnerabilities
discovered in five operators; however, more problems and vulnerabilities may be
present in these and other operators. As more and more operators provide VoLTE
services, it is essential that more security analyses be conducted on VoLTE
networks.

