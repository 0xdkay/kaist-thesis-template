\begin{abstract}
Long Term Evolution (LTE) is becoming the dominant cellular networking
technology, shifting the cellular network away from its circuit-switched legacy
towards a packet-switched network that resembles the Internet. To support voice
calls over the LTE network, operators have introduced Voice-over-LTE (VoLTE),
which dramatically changes how voice calls are handled, both from user equipment
and infrastructure perspectives. We find that this dramatic shift opens up a
number of new attack surfaces that have not been previously explored. To call
attention to this matter, this paper presents a systematic security analysis.
 
Unlike the traditional call setup, the VoLTE call setup is controlled and
performed at the Application Processor (AP), using the SIP over IP. A legitimate
user who has control over the AP can potentially control and exploit the call
setup process to establish a VoLTE channel. This combined with the legacy
accounting policy (e.g., unlimited voice and the separation of data and voice)
leads to a number of free data channels. In the process of unveiling the free
data channels, we identify a number of additional security problems of VoLTE
implementations, which lead to serious exploits, such as caller spoofing,
over-billing, and denial-of-service attacks. We identify the nature of these
vulnerabilities and concrete exploits that directly result from the adoption of
VoLTE. We also propose immediate countermeasures that can be employed to
alleviate the problems. However, we believe that the nature of the problem calls
for a more comprehensive solution that eliminates the root causes at mobile
devices, mobile platforms, and the core network.
%Finally, we present two fundamental solutions to the security issues of VoLTE.
\\
\\
\\
\keywords{VoLTE, Accounting, Security, Cellular Networks}
\end{abstract}
